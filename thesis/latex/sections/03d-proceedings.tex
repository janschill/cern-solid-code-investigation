\section{Proceedings in the CERN-Solid Collaboration}

With the two prototypes developed, integrated into Indico, and deployed to a running instance and showing a working solution with Solid principles in Indico, how can \gls{cern} proceed with its collaboration attempt? In the following the two fields of Solid specification implementations and Solid apps shall be looked at and debated how \gls{cern}'s future with Solid can look like.

\subsection{Servers}

The in \cite{cern-solid-investigation-spec} identified Solid implementations remain where they were at the time of composing the research paper \cite{cern-solid-investigation-spec}. A lot of development has happened for the \gls{css}, but it is still in a minor release version and therefore not yet deployed and switch out with \gls{nss} on the public solidcommunity.net domain. This is expected to happen as soon the developers trust the state of \gls{css} to facilitate at least the functionality of \gls{nss} and the tooling is in place to transfer all existing data pods currently hosted on the web server. No publicly communicated date has been set in stone for this to take place, but can be assumed to happen this year or next. The Solid specification on the other hand have a clearly defined completion date of 30.06.21 \cite{solid-tr}. Once the technical reports are completed -- but of course remain in the state of a living standard -- it can be expected to ease development process in the Solid ecosystem, as no major changes in the specification mean no critical new features need to be developed or supported by the server developments.

The \gls{nss} is maintained by a small team of unfunded open-source developers and only has capacity to fix major bugs and keep the dependencies up-to-date. It is still the most used Solid implementation and recommended data pod solution in conjunction with one of the providers, namely solidcommunity.net \cite{solid-community}.

\gls{cern} has a number of options in its proceedings with Solid servers in their current status.

\begin{enumerate}
    \item Outstanding solution through solidcommunity.net
    \item Integration with CERNBox
    \item Sandboxed \gls{css}
    \item Develop own server solution
\end{enumerate}

\paragraph{Outstanding Solution}\mbox{}\\

The usage and testing of the \glspl{poc} required a WebID and a data pod. Through extensive research and careful considerations it was concluded and recommended to all \gls{poc} participants to obtain the necessities through solidcommunity.net. The Solid Community is currently the most attractive data pod provider through its physical hosting in the \gls{uk} and openness regarding data usage and usage of \gls{nss} the go-to open-source Solid server solution. With the exit of the \gls{uk} from the \gls{eu} it seems uncertain what happens to those data residing in storage facilities in the country, which is by policy regulation for \gls{cern} not good enough, as all data need to reside in the \gls{eu}, abide to principles from \gls{oc11} \cite{oc11} and  \gls{gdpr} \cite{gdpr}, and that all data are merely used to provide its services \cite{policy-cern-server}. This is also a reason why a hosted instance through Inrupt \cite{inrupt} not acceptable. Not to mention the expected cost of such a service. An outstanding solution also gives away control over the running version of Server implementation. As of now most data pod providers run the \gls{nss}, but might switch over to \gls{css} -- which is desirable, but could bring new challenges. Even though a server implementation can only be called a Solid server when it adheres to the Solid specification, which can be tested by an \gls{ists} \cite{solid-test-suite}, which also means when the specification is followed features should be equally supported and implementations should allow interoperability. This has been proven to be difficult when looking at the several browsers, which are all implementing the \gls{http}, \gls{html}, and \gls{uri} standards (and many more specifications), but are all behaving slightly differently. This might be due to missing resources to stay up to date with feature development, or contrasting interpretations of the specifications. These risks will always endure and hence bring challenges.
\vspace{0.5cm}
\paragraph{Integration With CERNBox}\mbox{}\\

\gls{cern} uses CERNBox \cite{cernbox} to provide personal cloud storage to all \gls{cern} users to host and share files. The service is based of ownCloud \cite{owncloud} and hosted on \gls{cern} premise. In 2016 Nextcloud \cite{nextcloud} was formed from a fork of the open-source core software of ownCloud. PDS Interop \cite{pds-interop}, a collective of open-source developers, has developed a Nextcloud \cite{nextcloud} plugin to make the file hosting service Solid compatible. A Nextcloud server with the Solid plugin enabled passes currently the complete \gls{ists}. The integration with a running Nextcloud instance is as easy as installing the plugin through the web interface of Nextcloud.

An integration with CERNBox seems to be a suitable candidate, but also brings high risk. Adding the Solid implementation in to its own cloud storage infrastructure means the CERNBox administrators now also have to administer the Solid Nextcloud plugin and all the added complexity it enables. It would also be enabled for thousands of users adding even more area for possible deterioration.
A possibility could be to only enable it for a subset of users to test the integration on trained personnel. These are all resources that have to be accounted for. Most importantly though it would have to be analyzed how the attack surface is increased by enabling a plugin which includes an extra sharing functionality.
\vspace{0.5cm}
\paragraph{Sandboxed \gls{css}}\mbox{}\\

A less risky solution would be to use a sandboxed Solid implementation, such as \gls{css}. Once \gls{css} is released under a major version it could be deployed to self-contained OpenStack VM instances and then run as a new file storage system. This might seem redundant, considering \gls{cern} is already running a cloud storage system with CERNBox, but the added risk is much lower to infiltrate an existing system. The new \gls{css} deployments could solely be used for Solid related file storage and sharing and then incrementally be more integrated with existing \gls{cern} solutions. Even though the complexity might not be as high as with the installation through the Nextcloud plugin in CERNBox it still requires administrative work to keep the deployments running and updating \gls{css} regularly. Further, to even add more value and usability to this integration the \gls{css} could be used with the \gls{cas}. Meaning, instead of identifying with an external \gls{idp}, the \gls{cern} personnel could authenticate with their existing \gls{cern} accounts using \gls{cas}, which would need to be extended to also provide WebIDs.

Open-source and by the community owned software brings a lot of advantage, but also needs to be tread lightly. Software as this always relies on independent developers to fix bugs, develop features, which will not happen as fast when using a product with \glspl{sla}. A motivator for \gls{cern} to get involved with the development of open-source Solid services.
\vspace{0.5cm}
\paragraph{Own Server Solution}\mbox{}\\




\subsection{Applications}
